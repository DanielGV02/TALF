\documentclass[11pt]{article}
    \title{\textbf{Practica 1}}
    \author{Daniel García Villodres}
    \date{}
    
    \addtolength{\topmargin}{-3cm}
    \addtolength{\textheight}{3cm}
\begin{document}

\maketitle
\thispagestyle{empty}

\section*{Actividad 1}
Find the power set R3 of R = {(1, 1), (1, 2), (2, 3), (3, 4)}. Check your an-
swer with the script powerrelation.m and write a LATEX document with the
solution step by step.

\subsection*{}

Dado que $R^n = \lbrace(a,b):\exists x \in A, (a,x) \in R^{n-1} \land (x,b) \in R \rbrace$ entonces $R^3 = \lbrace(a,b):\exists x \in A, (a,x) \in R^{2} \land (x,b) \in R \rbrace$ \\
Como necesitamos R² para calcular R³, procedo a hallarlo.\\
$R^2 = \lbrace(1,1),(1,2),(1,3),(2,4)\rbrace$ \\
Por tanto, $R^3 = \lbrace(1,1),(1,2),(1,3),(1,4)\rbrace$




\end{document}

