\documentclass[fleqn, 10pt]{article}


\usepackage[utf8]{inputenc}
\usepackage{amsthm, amsmath}
\usepackage{nccmath} %Para centrar ecuaciones
\usepackage{graphicx}
\usepackage{enumitem}

    \title{\textbf{Practica 1}}
    \author{Daniel García Villodres}
    \date{}
    
    \addtolength{\topmargin}{-3cm}
    \addtolength{\textheight}{3cm}
    

    
\begin{document}

\maketitle
\thispagestyle{empty}

\section*{Actividad 2}
Consideremos $L=\{w\in \{a,b\}^* : w \textnormal{ no termina en } ab\}$. Un expresión regular que genera L es: \\

\subsection*{}

\begin{ceqn}
	\begin{align*}
	(a+b)^*a+(a+b)^*bb+a^*
	\end{align*}
\end{ceqn}
Diferenciamos 3 lenguajes: $L((a+b)^*a)$, $L((a+b)^*bb)$ y $L(a^*)$\\
El primero de ellos nos representa las cadenas formadas con a y b, terminadas siempre en a (con ello conseguimos las cadenas terminadas en aa y ba).\\
La segunda expresión representa el lenguaje de todas las cadenas formadas por a y b, terminadas en bb.\\
Por tanto, ya hemos conseguido todas las cadenas posibles que no terminen en ab, sin embargo, nos falta la cadena vacía $\epsilon$. Para ello, añadimos otro lenguaje como $a^*$ que contiene la cadena vacía.\\
La unión (+) de los 3 lenguajes nos dará la solución a la actividad.\\



\end{document}