\documentclass[fleqn, 10pt]{article}


\usepackage[utf8]{inputenc}
\usepackage{amsthm, amsmath}
\usepackage{nccmath} %Para centrar ecuaciones
\usepackage{graphicx}
\usepackage{enumitem}
\graphicspath{ {} } 

    \title{\textbf{Practica 2}}
    \author{Daniel García Villodres}
    \date{}
    
    \addtolength{\topmargin}{-3cm}
    \addtolength{\textheight}{3cm}
    

    
\begin{document}

\maketitle
\thispagestyle{empty}

\section*{Apartado 1}
Definición matemática del lenguaje que contiene la cadena a

\subsection*{}
Entre los estados $q_0$(inicial) y $q_1$(final) habrá una transición con la a, que será la única cadena que acepte. El estado $q_0$ tendrá también una transición con la b hacia $q_2$ en cuál será un estado "basura" en el que mandaremos aquellas cadenas que no vamos a aceptar. Lo mismo haremos desde $q_1$ con a y b. Por último en $q_2$ habrá una transición tanto con a como con b hacia sí mismo. La expresión matemática del autómata nos quedará así:\\
\\
$M=(K, \Sigma, \delta,s, F) = (\{q_0,q_1,q_2\}, \{a,b\}, \delta, q_0, \{q_1\})$

\begin{table}[h!]
\begin{tabular}{c|c|c}
  $\delta(q,\sigma)$ & $a$ & $b$\\
  \hline
  $q_0$& $q_1$ & $q_2$\\
  \hline
  $q_1$& $q_2$ & $q_2$\\
  \hline
  $q_2$& $q_2$ & $q_2$
\end{tabular}
\end{table}


\section*{Apartado 2}
Hacer el autómata en JFLAP e incluir la imagen en el documento

\subsection*{}
\includegraphics[scale=0.5]{captura}


\section*{Apartado 3}
También hacerlo en Octave, describiendo el JSON dentro de un entorno "verbatim" de Latex.

\subsection*{}
\begin{verbatim}
[
   {
    "name" : "a",
    "representation" : {
      "K" : ["q0", "q1", "q2"],
      "A" : ["a", "b"],
      "s" : "q0",
      "F" : ["q1"],
      "t" : [["q0", "a", "q1"],
             ["q0", "b", "q2"],
             ["q1", "a", "q2"],
             ["q1", "b", "q2"],
             ["q2", "a", "q2"],
             ["q2", "b", "q2"]]
      }
  }
]
\end{verbatim}

\end{document}